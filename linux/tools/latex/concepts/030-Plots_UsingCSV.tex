% Description: Plot Data from CSV Files

% Note
% 1. Use pgfplots package for plotting data directly from .csv files.
% 2. Select a column by the actual name from the .csv file using table[x=column 1,y=column 2...]

% Note on the CSV file
% 1. The CSV file must have comma as the column seperator.
% 2. The CSV file must have newline as the row seperator.
% 3. The column seperator and row seperator can be modified in the CSV files as well as in the code.
% 4. The numbers contained are actually longer than what shows up in the table because of rounding.

\documentclass{article}

\usepackage{siunitx}
\usepackage{tikz}                                                       % To generate the plot from .csv
\usepackage{pgfplots}

\pgfplotsset{compat=newest}                                             % Allows to place the legend below plot
\usepgfplotslibrary{units}                                              % Allows to enter the units nicely

% Setup siunitx:
\sisetup{
  round-mode          = places,  % Rounds numbers
  round-precision     = 2,       % to 2 places
}

\begin{document}

\begin{figure}[h!]
    \begin{center}
        \begin{tikzpicture}
            \begin{axis}[
                width=\linewidth,                                       % Scale the plot to \linewidth
                grid=major,                                             % Display a grid
                grid style={dashed,gray!30},                            % Set the style
                xlabel=X Axis $U$,                                      % Set the labels
                ylabel=Y Axis $I$,
                x unit=\si{\volt},                                      % Set the respective units
                y unit=\si{\ampere},
                legend style={at={(0.5,-0.2)},anchor=north},            % Put the legend below the plot
                x tick label style={rotate=90,anchor=east}              % Display labels sideways
                ]
                \addplot
                % Add a plot from table.
                % Select the columns by using the actual name in the .csv file (on top)
                table[x=Column 1,y=Column 2,col sep=comma] {030-Plots_Data.csv};
                \legend{Plot}
            \end{axis}
        \end{tikzpicture}
        \caption{Autogenerated plot.}
    \end{center}
\end{figure}

\end{document}