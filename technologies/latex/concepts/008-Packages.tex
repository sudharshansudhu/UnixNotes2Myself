% Description: Packages in LaTeX

% Note
% 1. Packages add new functions to LaTeX.
% 2. All packages must be included in the preamble.
% 3. Packages add features such as support for pictures, links and bibliography etc.
% 4. To import a package in LaTeX, add the \usepackage{PACKAGENAME} directive to the preamble of a document.
% 5. When using Linux or Mac, most packages will already be installed by default. In case of Ubuntu installing
%    texlive-full from the package manager would provide all packages available.
% 6. Example: An example of package to typeset math is an environment called equation. Everything inside this
%    environment will be printed in math mode, a special typesetting environment for math.

\documentclass{article}

\usepackage{amsmath}

\begin{document}

\begin{equation}
    (x) = x^2
\end{equation}

% Note
% 1. The equation environment does not need amsmath package but it numbers each of the equations. The automatic
%    numbering is a useful feature, but sometimes it's necessary to remove them for auxiliary calculations.
% 2. LaTeX doesn't allow this by default but through a package amsmath included above.
% 3. Use equation* environment instead of equation environment to disable auto-numbering.
\begin{equation*}
  f(x) = x^2
\end{equation*}

\end{document}